\documentclass[a4paper ,12pt]{article}
\usepackage{graphicx}
\newcommand*{\plogo}{\fbox{$\mathcal{DU}$}} 
\newcommand*{\titleGP}{\begingroup
\centering 
\vspace*{\baselineskip} 
\rule{\textwidth}{1.6pt}\vspace*{-\baselineskip}\vspace*{2pt} 
\rule{\textwidth}{0.4pt}\\[\baselineskip] 
{\LARGE IDENTIFYING STAINS}\\[0.2\baselineskip] 
\rule{\textwidth}{0.4pt}\vspace*{-\baselineskip}\vspace{3.2pt} 
\rule{\textwidth}{1.6pt}\\[\baselineskip] 
\scshape % Small caps
An Investigation into Potential Methods of Measuring Composition and Quantity of Stains on Hard and Soft Surfaces \\ 
[\baselineskip]Durham University Physics Department, Epiphany 2014\par 
\vspace*{2\baselineskip} 
{\Large Prof. A. Monkman \\ J.Armstrong \\ D.Boddington \\ B.Corteling \\ S.Crossland \\ C.Pei\par}
{\itshape Durham University\\ United Kingdom\par} 
\vfill 
\plogo \\[0.3\baselineskip] 
{\scshape\LARGE 2012} \\[0.3\baselineskip] 
\endgroup}



\begin{document}
\pagestyle{empty}
\titleGP

\newpage
\begin{abstract}
Abstract in here
\end{abstract}

\section{Introduction}

\subsection{Motivation}

\subsubsection{Procter \& Gamble}
This section needs to detail P\&G's reasons for conducting this investiagtion and possibel outcomes ie advertising stronger brand recognition

\section{Current Methods}

This section needs to detail the current industry standards of measuring stain removal on hard and soft surfaces. ie LAB colour and mass difference- theory and current P\&G protocol. Comment on assessment of Removal and Redeposition. protocol can be in appendices.

\section{Fats \& Removal}
\subsection{Triglycerides}
Describing the composition of fats from long chain acid and glycerides - ester bond.\\Insert some pretty pictures here
\subsection{Detergents}
Detailing the way in which detergents work to break down fats during a washing cycle.\\Insert some pretty pictures here

\section{Aims}
\subsection{Composition of Stains on Hard \& Soft Surfaces}
Detailing the brief from Natalie
\subsection{Quantifying Stains on Hard \& Soft Surfaces}
Detailing the brief from Nazar

\section{Potential Methods of Investigation}
\subsection{Electromagnetic Resonance James}

Nuclear Magnetic Resonance (NMR) is a phenomenon observed as a consequence of the differing magnetic properties of varying isotopes of atomic nuclei. Nuclei in the presence of a magnetic field absorb and re-emit electromagnetic radiation. The wavelength of the emitted photon is dependent upon the specific properties of the atom in question, and thus is a powerful tool for identifying elements experimentally. As the resonance of a substance is proportional to the strength of the applied magnetic field it is possible to create high resolution images of a sample by placing it in a non-uniform field, and then measuring the Zeeman effect and the Knight shift. This is done first by measuring the alignment of the magnetic nuclear spins, and then perturbing these spins with electromagnetic radiation.\\\\NMR has become a widely used tool for the analysis of fatty acids. Previously, the most predominant method for determining fatty acid composition was via gas chromatography (GC)\footnote{Craske, J.D., Separation of Instrumental and Chemical Errors in the Analysis of Oils by Gas Chromatography $-$ A Collaborative Evaluation, J. Am. Oil Chem. Soc. 70 pp. 325$-$334 (1993).}. However GC requires lipids and oils to be converted into methyl esters before being analysed\footnote{Watanabe, S.,T. Nagai, S. Hayano, K. Ikelda, A. Kato, K. Kumihiro, T. Murui, and M. Negishi, Determination of Fatty Acid Composition by Programmed Temperature Gas Chromatography, Yukagaku 24 pp. 588-594 (1975)}.  This method is also very time-consuming, with approximately a 30 minute delay between observations. Over the last twenty years NMR has been used as the basis for several reports on the composition of oils, and is the one of the most promising new methods for determining organic structures\footnote{Sacchi, R., T. Medina, S.P. Aubourg, F. Addeo, and L. Palillo, Proton Nuclear Magnetic Resonance Rapid and structure-Specific Determination of $\omega-$3 Polyunsaturated Fatty Acids in Fish Lipids, J. Am. Oil Chem. Soc. 70: pp. 225$-$228 (1993).}.\\\\In his 1983 paper, Shoolery showed methods for determining the composition of vegetable oils using $^{13}$C NMR\footnote{Shoolery, J.N., Applications of High Resolution Nuclear Magnetic Resonance to Study of Lipids, in Dietary Fats and Health, edited by E.G. Perkins, American Oil Chemists’ Society, Champaign, 1983, pp. 220$-$240.}.  The aptitude of this method for determining the composition of fatty deposits on textiles is in good agreement with the brief for this project, as this technique is able to determine the levels of saturation in triglycerols, and is used in addition to $^1$H NMR in order to determine constituent molecules in a mixture of fatty acids\footnote{Johnson, L.F., and Shoolery, J.N. (1962) Determination of Unsaturation and Average Molecular Weight of Natural Fats by Nuclear Magnetic Resonance, Anal. Chem. 34, 1136$-$1139}.
\\\\
\begin{figure}
\centering
\includegraphics[height= 150pt, keepaspectratio=true, draft=true]{test_pic.jpg}
\end{figure}\\\\

In order to implement a NMR spectroscopy facility, there are several key factors to take into account. Firstly, the process is very sensitive to fluctuations in the environmental parameters in the locale of the equipment. These parameters include vibrations, variations in air temperature, and stray electromagnetic radiation. It would be necessary to undertake a survey of the site in order to assess its suitability for such experimental equipment. Secondly, the use of high magnetic field strengths causes issues with regards to both the effect on other nearby equipment, and the precautionary protocols which must be undertaken with respect to personnel health.\\\\Due to the high costs of using this method in our analysis, and the difficulties arising when building a suitable facility for the continued use of this method at P\&G it has been discounted.




\subsection{Fluorophore Emission}
\subsection{Hyperspectral Imaging}
\subsection{Infrared Spectroscopy Echo}
Infrared spectroscopy is an analytical technique that used to recognize the structure of molecules from their characteristic absorptions in IR region\footnote{Coates, John. "Interpretation of infrared spectra, a practical approach." Encyclopedia of analytical chemistry (2000).}. The frequencies of absorptions correspond to the vibration of certain chemical bonds in the molecules. Actually, a molecule can have infinite number of vibrational modes while here we consider the modes that described in a threefold set of axes which are called normal modes of vibration. Those vibrations are divided into two parts, stretching and bending. In practice, most vibrations have almost the same energy and here we only look at the frequencies in infrared region. When the infrared radiation passes through the sample, radiation with the same frequency of vibration is absorbed. This technique is widely used in organic Chemistry to recognize the structure of a molecule. By comparing with the spectrum of known molecules, the composition of compounds can be determined\footnote{Duckett, Simon, and Bruce Gilbert. Foundations of spectroscopy. New York: Oxford University Press, 2000.}. 
The absorption spectrum are very complicated for organic molecules as the number of vibrational modes increase rapidly with the increasing of molecule size. However, it is still possible to examine the composition of compound from the Infrared spectroscopy because certain absorptions can be associated with stretching or bending vibrations of particular bonds. For example, some special functional group such as C$-$H and C$=$O can be recognized and the structure of molecule can be confirmed by the transition frequency of those special functional groups. And the basic components of grease are glycerides, fatty acids, protein and carbohydrates, and the infrared spectroscopy of those chemicals were measured in a series of papers\footnote{O'Connor, Robert T., Elsie F. DuPr\'{e}, and R. O. Feuge. "The infrared spectra of mono-, di-, and triglycerides." Journal of the American Oil Chemists Society 32.2 (1955): 88-93.} \footnote{Sato, Tetsuo, Sumio Kawano, and Mutsuo Iwamoto. "Near infrared spectral patterns of fatty acid analysis from fats and oils." Journal of the American Oil Chemists Society 68.11 (1991): 827-833.}. Thus, it is possible to determine the composition of grease by infrared spectroscopy.
Since the fluorescent agent in detergent cannot be excited in infrared region, infrared spectroscopy has relative low background noise. However, the absorption spectrum need to be obtained by passing infrared beam through the sample, [1] which means the sample has to be relative transparent while most of our samples are non-transparent\footnote{Stuart, Barbara. Infrared spectroscopy. John Wiley \& Sons, Inc., 2004.}. Thus, infrared spectroscopy is not an ideal solution to our problem.

\subsection{Radioactivity Sarah}
Doping a sample radioactively enables the detection of alpha, beta or gamma ray emission, dependent of the source used to dope the grease stain, when the nucleus of the specified atom decays\footnote{http://terpconnect.umd.edu/$\sim$alaporta/PHYS271$\_$s09/Phys271-Exp8-rev.pdf accessed at 18:39 on 23.02.2014}. If an initial quantity of radioactive material is specified for insertion into the grease sample and the known half-life for the doping agent known, then we can predict the expected loss via decay and deduct this from the radioactive material lost in the grease which is removed via washing.  This in theory could provide us with an effective way of quantifying the grease removal, but there are numerous flaws in both the method itself and safely implementing the procedure in a work place environment that do not make it a viable option.\\\\Before any testing can be done via this method, the correct safety protocol for handling radioactive material must be adhered to.  All sources must be handled at appropriate distances from the body, skin contact removed and shielding around the source added\footnote{http://ehs.unl.edu/sop/s-sealed$\_$sources.pdf accessed at 19.16 on 23.02.2014}. The sources must also be kept in heavily sealed and monitored conditions to reduce radioactive emissions into the surrounding area and everyone must be made aware of the full safety protocol and trained in this also [2]. Implementing this level of safety procedure would not be difficult; the difficulty comes in the cost and possible risks this method causes. Shielding and storage are expensive and require additional room to be created to house such things, alongside the large risk to safety of handling radioactive material as frequently as needed this method has obvious limitations of use in a chemical manufacturing and testing setting.\\\\Radioactive decay is itself a random process, so although a mean life time can be calculated it does not provide a particularly accurate measurement of calculating the count rate at any one time [1]. Although we can measure the detected rate with a Geiger counter in the pre and post wash samples to estimation of the decays lost without grease removal is not accurate enough to be able to quantify our grease stain as well as other methods.\\\\There may also be difficulty in doping the grease stains with the radioactive material. We need to ensure and even spread across the inhomogeneous stain in order to quantify the amount of grease we have, but finding a way to put the radioactive sample in the stain in such a way is almost impossible to find. Complex procedures to do this could be procured, but ease of mixing with the grease needs to considered and the process needs to be repeatable on a large scale with little error, which is difficult.\\\\Quantifying grease could work by this method if it was possible to conduct across Proctor and Gamble, however it would not provide us with an understanding of the constituents of the stains we are examining before and after washing. Other methods could be found that would enable us to test composition and quantify the stain with one piece of apparatus, which would reduce: costs, test timings, health and safety procedures that need to be adhered to and be easier to roll out across a large workforce with reduced training required. 

\subsection{Raman Scattering Echo}
Raman spectroscopy is another common method to detect the vibrational modes of molecules beside infrared spectroscopy.Since the discovery of Raman scattering in 1928, it has become a powerful method to detect the instinct properties of compounds\footnote{Opilik, Lothar, Thomas Schmid, and Renato Zenobi. "Modern Raman Imaging: Vibrational Spectroscopy on the Micrometer and Nanometer Scales." Analytical Chemistry 6 (2013): 379-394. }. It has been successfully applied on the analysis of triglycerides and proteins\footnote{Rubio‐Diaz, Daniel E., and Luis E. Rodriguez‐Saona. "Application of Vibrational Spectroscopy for the Study of Heat-Induced Changes in Food Components." Handbook of Vibrational Spectroscopy (2010).}\footnote{Bresson, S., M. El Marssi, and B. Khelifa. "Conformational influences of the polymorphic forms on the CO and C–H stretching modes of five saturated monoacid triglycerides studied by Raman spectroscopy at various temperatures." Vibrational spectroscopy 40.2 (2006): 263-269.}. Imaging of complicated sample such as tissues and cells with high resolution is also realized via Raman spectroscopy\footnote{Evans, Conor L., et al. "Chemical imaging of tissue in vivo with video-rate coherent anti-Stokes Raman scattering microscopy." Proceedings of the National Academy of Sciences of the United States of America 102.46 (2005): 16807-16812.}\footnote{Nan, Xiaolin, Eric O. Potma, and X. Sunney Xie. "Nonperturbative chemical imaging of organelle transport in living cells with coherent anti-stokes Raman scattering microscopy." Biophysical journal 91.2 (2006): 728-735.}.  
Here the vibrational transition is measured by the energy shift of scattered photons in inelastic scattering. In most cases, incoming photons are elastically scattered by molecules and scattered photon have same frequencies with incident ones. [1] This kind of elastic scattering is called Rayleigh scattering. However, in a few cases, the incoming photons interact with target molecules and change their vibrational states.[1] The frequency of photon are also shift the same amount as the energy change of molecules during this process. There are two types of inelastic scattering, stokes scattering and anti-stokes scattering. In stokes cases, the molecule is in its ground state initially, and in an excited state after this process. In anti-stokes cases, the molecule is in an excited state but returns to the ground state after the interaction. The spectrum of CCl$_4$ in 488nm laser is demonstrated in figure 1.\footnote{A RAMAN lecture pdf} The strongest signal in the middle of the spectrum is Rayleigh scattering. Stokes scattered photons are those lose energy while anti-stokes photons are those gained energy during scattering. It can be clear seen that, the frequencies of two types of scattered photons are exactly symmetry while stokes scattering are much stronger that anti-stokes scattering because in room temperature, most molecules are in their ground states.[1]\\\\Comparing with IR spectroscopy, an important advantage of Raman scattering is the possibility to determine the quantity of each component for both transparent and non-transparent samples. While in IR spectroscopy, only transparent sample is acceptable. Furthermore, there is no special preparation of sample is need in Raman scattering, the grease sample can be used in Raman scattering directly. [5] 
Comparing with others…….\\\\There are also several limitations in Raman scattering. One of them is that the laser used to obtain Raman spectrum is in visible region while some grease shows strong fluorescent effect in visible region since some proteins in the grease contain chromophores. There are several possible ways to reduce the fluorescent effect. One of them is extended photo bleaching, which shows considerable effectiveness to reduce the fluorescent background\footnote{Macdonald, A. M., and P. Wyeth. "On the use of photobleaching to reduce fluorescence background in Raman spectroscopy to improve the reliability of pigment identification on painted textiles." Journal of Raman Spectroscopy 37.8 (2006): 830-835.}. [8] However, this method is very time consuming, since it takes quite a long time to bleach every single point.  Another one is to use laser with lower frequency, because the fluorescent effect should be weaker in low frequency. Moreover, it is difficult to measure the grease removal on fabric via Raman spectroscopy because the frequency range of fabric and grease are almost overlap. Although there are certain limitations for Raman spectroscopy, it is still a relative effective way to determine the grease composition among other methods.  In our experiment, we use a confocal Raman microscopy system to obtain the Raman spectrum of the compound and the structure of it is shown in figure 2. In this system, the laser beam is focused onto the sample surface by a microscope objective and the scattered light is collected by the same objective.[1]\\\\Both the incoming laser and the scattered radiation need to pass through a small pinhole, which means only the scattering on the focal spot are pass the pinhole. Since only the light that scattered by can pass through the pinhole, confocal microscopy can scan the sample in 3 dimensions. Then the Rayleigh scattering is removed by a band-pass filter, and only inelastic scattered photon are reserved. Although both stokes and anti-stokes signal are used to generate the spectrum, the spectrum is mainly contribute by stokes scattering.  
Technically, the information of composition is contained in the spectrum. Since our grease is the mixture of many different type of compounds, the obtained spectrum is the superposition of all the components. By comparing the spectrum with known compounds to identify the main components in the sample. 

\subsection{Surface Reflectance}
\subsection{Ultraviolet Absorption Sarah}
Ultraviolet absorption spectroscopy regards the absorption of high-energy light, which causes electronic excitation within organic compounds. Between the wavelengths of 200-800 nm absorption can be shown\footnote{http://www2.chemistry.msu.edu/faculty/reusch/virttxtjml/spectrpy/spectro.htm accessed at 22:08 on 23.02.2014}.\\\\UV spectroscopy will show peaks in different bands dependent on the nature of the bonds present within the grease, however if there are impurities which are non-organic which have infiltrate the grease stain applied then these would not show ultraviolet absorption\footnote{http://www.chemguide.co.uk/analysis/uvvisible/theory.html accessed at 22:18 on 23.02.2014}. Constituents of the stains are easily determined due to the different excitation of electrons within different conjugated electron-pi systems and hence the different wavelengths of light emitted from the different bonds [1]. However numerous area of the stains surface would need to be covered at regular intervals due to the inhomogeneity of the grease. Although this method can return which type of bonds are present and allow us to identify groups which are present; such as carboxylic acids, it would be difficult to distinguish the exact compounds presents using this method.\\\\Quantifying the stains will also prove difficult using this method so an alternative may need to be found for this, meaning that further tests would need to be conducted on the stains after the aforementioned spectroscopy which would add additional time constraints to sample testing.\\\\Another possible flaw in this method is the use of optical brighteners within modern detergents. This will not restrict the testing of samples before washing, but afterwards may have a significant effect on the results taken from the post wash set of fabric grease stains. Optical brighteners work by attaching to the cloth during washing, via UV absorption they emit light in the blue range of the spectrum and can make clothes which have faded over time appear whiter than they actually are\footnote{http://www.dispersions-pigments.basf.com/portal/basf/ien/dt.jsp?setCursor$=$1$\_$556358 accessed at 22:36 on 23.02.2014}. The presence of these within a grease stain sample can cause a distortion of the actual results collated from the organic materials themselves, therefore posing a flaw in this method of determining the constituents of a sample. The natural fluorescence of the fat could also contribute to a masked reading for the bonds present. 

\subsection{X-Ray Spectroscopy James}
X-rays have frequencies in the range 3$\times$10$^{16}$Hz to 3$\times$10$^{19}$Hz, and have energies of the keV order. X-rays are emitted during the electronic transitions to an inner shell of the atom. The frequency of the emitted x-ray is characteristic of the atom type, and therefore different elements have different x-ray spectra. This means that x-ray spectroscopy is a powerful diagnostic tool for identifying the chemical composition of a sample material.
In the simple model of a hydrogen-like atom, electronic transitions between an initial state n$_{initial}$ and a final, lower state n$_{final}$, lead to a photon being emitted with energy\footnote{Eisberg, R. and Resnick, R., Quantum Physics of Atoms, Molecules, Solids, Nuclei, and Particles, 2nd Edition, (1985)},\\\begin{displaymath} E_{\gamma} = E(n_{i})-E(n_{f}) =13.6eVZ^2\left(\frac{1}{n^2_i}-\frac{1}{n^2_f}\right) \end{displaymath}
However, in more complex atoms, such as those which would be under scrutiny in the course of this investigation, we introduce a screening factor such that this relationship may still hold. This correction accounts for the fact that the electrons in the outer shell act as if they were screened from the full charge of the nucleus by those electrons in lower lying shells.,\\\begin{displaymath} E_{\gamma} = E(n_{i})-E(n_{f}) =13.6eV(Z-s)^2\left(\frac{1}{n^2_i}-\frac{1}{n^2_f}\right) \end{displaymath}\\\\In the case of the n=2 to n=1 transition, if a plot of the square root of the energies for several elements is made against the atomic number, Z, of the element, a linear relationship is found. The gradient of the curve is proportional to the ionisation energy of hydrogen, and the intercept is related to the screening factor. Once these calibration measurements have been made, the atomic number of any sample element can be deduced from its position on the curve. Similar calibrations may be made from other electronic transitions.\\\\These transitions may be induced by either being excited by high energy particles, a method known as particle induced x-ray emission. This is typically carried out with an alpha source. However, another common method involves exciting the sample with another source of x-rays, known as x-ray fluorescence\footnote{Melissinos, A. and Napolitano, J., Experiments in Modern Physics, 2nd Edition Academic Press; (March 31, 2003)}.\\\\Co-57 decays to Fe-57, and in doing so releases a photon of the energy order 14-122 keV\footnote{Knoll, G., Radiation Detection and Measurement, 3rd Edition (2010)}.   However, due to the nature of the radioactive material required for the pursuit of this method and the protocols which would have to be enforced for its use in the corporate work place, it has been discounted from our investigation.


\subsection{3D Electron Microscope Sarah}
3D electron microscopes are a combination of the properties of: a scanning electron microscope (SEM), an optical microscope and a roughness gauge. By combining the properties of the aforementioned it is then possible to alleviate some of the disadvantage of each method when it comes to scanning and analysing complex products\footnote{http://www.keyence.com/products/microscope/laser-microscope/vk-x100$\_$x200/features/index.jsp accessed at 17:38 on 23.02.2014}.\\\\Using an optical microscope alone it is difficult to focus on a target with an uneven surface at high magnifications, such as the surface of the grease stains we are investigating on all surfaces.
A standard electron scanning microscope can only observe the intended target in black and white, but as the fat naturally fluoresces then colour may help to distinguish between patches of fluorescence or to optimise and enable us to spot with more clarity unexpected traces we did not expect to see within the stain. 3D electron microscopes also remove a lot of the preparation and observation time required to treat a sample in comparison to SEM observations. Roughness gauges causes damage to the target area when measuring projections and depressions on the product being tested, in our case grease stains. Overcoming such a barrier enables us to measure the sample more precisely and will provide us better results for quantifying the grease stains [1].\\\\
The clarity of images is also increased when using a 3D electron microscope in comparison to an SEM or optical microscope\footnote{http://www.keyence.com/products/microscope/laser-microscope/vk-x100$\_$x200/features/feature-02.jsp accessed at 17:45 on 23.02.2014}. Our grease stains contain great inhomogeneity across their surfaces and increased resolution allows for better observations of these regions and any impurities we may find within the stains. Our stains are inhomogeneous over a large area and to scan one point at a time would not able maximum time efficiency for measuring the sample. 3D electron microscopes can be chosen with a wide field view of 200mm $\times$ 125mm with a repeatability error of $\pm$2 microns with stage movements across the sample\footnote{http://www.keyence.com/products/microscope/laser-microscope/vk-x100$\_$x200/features/feature-05.jsp accessed at 18:12 on 23.02.2014}. It also has the ability to scan most surfaces with this level of precision, so it would work on our fabric, steel and glass/plastic samples which have been deposited with grease.\\\\The equipment itself comes alongside analysis software which enables the quick calculation of the area covered in three dimensions by the grease deposit on our test material. This would reduce the time taken for us to analyse the samples and provide us with a set of results and errors from simply inputting a sample \footnote{http://www.keyence.com/products/microscope/laser-microscope/vk-x100$\_$x200/features/feature-04.jsp accessed at 18:17 on 23.02.2014}. However the downsides to this is method are that we are only able to take a measurement to quantify the amount of grease covering the target area and the entire sample would need to be repositioned and scanned to cover all the area affected as the grease layer is not uniform. Time constraints will therefore strongly affect this method and a further method would need to be developed in order to test the constituents of the grease before and after washing. A far better method would be one that could incorporate both quantity and constituent measurements in one. The price is also a determining factor and a 3D electron microscope costs between \pounds29-36,000 exclusive of VAT and training. And while training is provided, along with technical support, further extensive training would need to be carried out internally within the company to all using the equipment which would be costly and time consuming, determining this method to not be particularly viable for use within a large workforce.

\section{Methods to be Investigated}
\subsection{Raman Spectroscopy Echo}
Raman spectroscopy is the relative better one among other composition methods, since the discovery of Raman scattering in 1928, it has become a powerful method to detect the instinct properties of compounds$^3$. It has been successfully applied on the analysis of triglycerides and proteins $^5$ $^6$. Imaging of complicated sample such as tissues and cells with high resolution is also realized via Raman spectroscopy$^7$ $^8$.  
[Comparing with other methods]I’ll finish it later since I need to compare it with other method.
The basic ‘procedure?? Protocol??’ : firstly, measure the Raman spectrum of grease sample before and after washing. Since our grease is the mixture of many different type of compounds, the obtained spectrum is the superposition of all the components. Secondly, comparing the spectrum with known compounds to identify the main components in the sample. Finally, calculate the grease removal percentage by comparing the relative intensities of the peaks.(just a draft since we have really start the experiment yet)

\subsection{Fluorophor Emission}
\newpage

\section{Methodology \& Results}
Some writing.\\
\subsection{Raman Scattering on Hard Surfaces}
Beginning\\
\begin{figure}[h]
\centering
\includegraphics[height= 150pt, keepaspectratio=true, origin=c, draft=true]{5MetalSlides.jpg}
\caption[\small Figure 1.]{\small{Hard Surfaces (Stainless Steel) with baked on grease that has been subsequently run through cleaning protocol}}
\end{figure}
\\Some more writing to compare the font size of the main body of text to the caption font size
\subsection{Analysing the Quantity of Oils on Glassware}
Some writing.\\
\begin{figure}[h]
\centering
\includegraphics[origin=c, angle =-90, height= 150pt, keepaspectratio=true, draft=true]{RG6Bottle.jpg}
\end{figure}
Even more writing.\\
\begin{figure}[h]
\centering
\includegraphics[height= 150pt, keepaspectratio=true, draft=true]{RG6GlassSlide.jpg}
\end{figure}
\subsection{Intrinsic Fluoroscent Properties of Fats}
\begin{figure}[h]
\centering
\includegraphics[origin=c, angle =-90, height= 150pt, keepaspectratio=true, draft=true]{9FabricStains.jpg}
\end{figure}
More text

\section{Discussion}
\subsection{Findings}
\subsubsection{Raman Scattering on Hard Surfaces}
\subsubsection{Analysing the Quantity of Oils on Glassware}
\subsubsection{Intrinsic Fluoroscent Properties of Fats}

\subsection{Limitations}
\subsubsection{Raman Scattering on Hard Surfaces}
\subsubsection{Analysing the Quantity of Oils on Glassware}
\subsubsection{Intrinsic Fluoroscent Properties of Fats}

\subsection{Further Areas Of Interest}




\section{Conclusion}


\newpage
\begin{center}
\rule{\textwidth}{1.6pt}\vspace*{-\baselineskip}\vspace*{2pt} 
\rule{\textwidth}{0.4pt}\\[\baselineskip]
{\LARGE \bfseries{APPENDICES}}\\[0.2\baselineskip]
\rule{\textwidth}{0.4pt}\vspace*{-\baselineskip}\vspace{3.2pt} 
\rule{\textwidth}{1.6pt}\\[\baselineskip]
\end{center}
\appendix
\section{Minutes}
\section{Stain Manufacture Protocol}
\section{L*a*b* Colour Model Protocol}
\section{Mass Difference Protocol}
\section{Glassware Protocol}





\end{document}
